Fresh students may find it hard to understand the concepts of object oriented programming.
Especially when faced with a model view controller architecture, as part of a graphical program, the internal connections can become complex.
In order to help them with their understanding it is desirable to investigate the use of tools that can help them understand both the structure and order of execution in a Java program.
This report will look into the available tools, and how they may fit into the teaching environment used at NTNU.
Additionally, it will look at possible improvements, and examine whether the tools are considered useful by the students.
Approach:
A multi-pronged approach was used: First, a collection of tools that fit the purpose was identified and judged against each other.
The most promising of these was then further examined, identifying possible areas for improvement.
Some of these suggested changes were then implemented.
After implementing, the tool and its modifications were evaluated by a small group of students, with the intention of determining the usefulness of such a tool as part of the education.
Results:
The introductory study concluded that the tool JIVE was the best fit for further study due to its features.
Upon further examination, several issues were found, of which some were improved.
The following evaluation resulted in a positive response from the participants, who could clearly see a use for such tools as part of the education.
The implemented changes were all considered to be improvements by the students, but they also pointed out further modifications.
Conclusions:
The results are promising, but there is need for additional work to properly determine its usefulness.
Further improvements regarding both performance and usability can be made.
User testing at a larger scale is necessary to properly determine the possible gains of integrating JIVE into the courses.