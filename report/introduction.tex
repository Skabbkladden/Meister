\section{Introduction}\label{introduction}
~\\
%stikkord
New students may find programming in general, and object oriented programming in particular, difficult.
The understanding of computers, and how they work, is not given a lot of focus in the school system, and only a few students have actually tried learning programming before coming to the university.
With little or no previous knowledge, it can be hard to understand the concepts, and to get a mental model of what is going on when writing a program.
Especially code not written by themselves, for example exercise frameworks, and code generated by various tools, can be hard to get a good understanding of.
Traditional debuggers are not necessarily helping when detecting a runtime error, and often significant amounts of time is spent searching for the cause of a bug, instead of actually fixing it\cite{ko2006}.
Tools that present the state of a program in a simple visual way may help to understand.
~\\

Learning goals for MMI-course (man-machine-interaction):\\
Introduce the student to concepts, methods and techniques for designing man-machine-interfaces, knowledge and skills in object oriented construction of graphical window-based interfaces.
~\\

Will such tools be useful in helping students to reach the learning goals of MMI and other beginner-courses?
Are existing tools good enough?
If not, can something be modified to better fit the purpose?