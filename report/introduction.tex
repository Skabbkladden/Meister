\chapter{Introduction}\label{introduction}

Upon starting their studies in computer science, new students may find the general concepts of programming, and object oriented programming in particular, to be difficult to grasp.
The understanding of computers, and how they work, is not given a lot of attention in the school system -- which is more concerned with the use of computers as tools, not how these tools are made -- and only a few students have actually tried learning programming before coming to the university. %TODO: CITE

With little, or no previous knowledge, it can be hard to understand the concepts, and to get a mental model of what is going on when writing a program.
Especially code not written by themselves, for example exercise frameworks, and code generated by various tools, can be hard to get a good understanding of.
Traditional debuggers are not necessarily helpful when detecting a runtime error, as they tend to only point at where the error occurred, along with a classification.
Depending on the language and tools being used, this error may be specific enough to, for example, indicate an indexing error, or so vague that it only says 'something went wrong here', often resulting in significant amounts of time being spent searching for the cause of a bug, instead of actually fixing it \cite{ko2006}. %hva er en debugger? hvilke metoder innebærer tradisjonell debugging.
The use of \glspl{ide} -- tools that integrate source code editors with facilities for compiling and running programs, and intercepting errors that occur during run-time -- can help with the challenge of finding the cause of an error, and might even suggest a way to fix it, but getting an overview of an entire program, and how its components interact with each other is not necessarily easier.
Introducing tools that present the state of a program in a simple, visual way may help students understand how a program works, and how its components interact with each other.

During the second year of the computer science, and the informatics studies at NTNU, there is an increased focus on projects and more complex software.
Among the mandatory courses of these studies, we find the course TDT4180~-~Human--Computer Interaction (HCI).
This course handles topics related to creating programs with a \gls{gui}, and specifically how to implement a \gls{mvc} architecture using Swing -- a part of the framework surrounding the Java programming language, providing components that are frequently used in graphical interfaces.
The learning goals for the HCI-course are as follows:
\begin{quote}
Introduction to important concepts, methods and techniques related to human-computer interaction and design of user interfaces.
Knowledge and practical experience with implementation of user interfaces in object-oriented frameworks.
\end{quote}

Through the course, students are introduced both to the theories of good user interface design, and how such interfaces can be implemented in Java.
Java, and the \gls{ide} Eclipse, make a common framework for many of the programming courses at NTNU, which makes compatibility with them an important criteria when looking for tools to support the students.

\section{Research Questions}\label{intro-RQs}

The goal of this report is to look into the available tools that can aid developers, concentrating on the potential use in the HCI-course.
This implies a focus on how to better understand the interactions within programs implementing the \gls{mvc} pattern, as well as the behavior of graphical interfaces.
Looking into the tools, it will identify the different types, as well as their strengths and weaknesses, using this as a basis for comparison.
After comparing, and determining which tool is likely to be of most use, the report will look at both the potential for improvements, and how it may fit into the teaching process, including the effect of any improvements made.
To be more specific, the following research questions have been formulated:
\begin{theorem}
What is the current state of the various visualization tools that are available?
\end{theorem}
\begin{theorem}
Could any of these be integrated into the current teaching environment at NTNU, consisting of Java and Eclipse? %slå sammen med 1?
\end{theorem}
\begin{theorem}
Is there room for improvement in how these tools are used, and the ease of using them? Can the information they provide be refined, or presented in a more understandable way? %presiser, nestten garantert nei nå
\end{theorem}
\begin{theorem}
Would the use of such tools and any improvements actually be useful for the students, and help them understand the internal interactions in a program?%på hvilken måte?
\end{theorem}
