\chapter{Introduction}\label{introduction}
~\\
Upon starting their studies in computer science, new students may find the concepts of programming in general, and object oriented programming in particular, to be difficult to grasp.
The understanding of computers, and how they work, is not given a lot of focus in the school system, and only a few students have actually tried learning programming before coming to the university.%TODO: CITE
With little or no previous knowledge, it can be hard to understand the concepts, and to get a mental model of what is going on when writing a program.
Especially code not written by themselves, for example exercise frameworks, and code generated by various tools, can be hard to get a good understanding of.
Traditional debuggers are not necessarily helpful when detecting a runtime error, and often significant amounts of time is spent searching for the cause of a bug, instead of actually fixing it \cite{ko2006}. %hva er en debugger? hvilke metoder innebærer tradisjonell debugging.
Tools that present the state of a program in a simple, visual way may help students understand how a program works, and how its components interact with each other.
~\\

During the second year of the computer science, and the informatics studies at NTNU, there is an increased focus on projects and more complex software.
Among the mandatory courses of these studies, we find the course Human-machine-interaction (HMI).
This course handles topics related to creating programs with a \gls{gui}, and specifically how to implement a \gls{mvc} architecture using Swing, a part of the framework surrounding the Java programming language, providing components that are frequently used in graphical interfaces.
The learning goals for the HMI-course are as follows:
\begin{quotation}
Introduction to important concepts, methods and techniques related to human-computer interaction and design of user interfaces.
Knowledge and practical experience with implemetation of user interfaces in object-oriented frameworks.
\end{quotation}

Through the course, students are introduced both to the theories of good user interface design, and how such interfaces can be implemented in Java.
Java, and the \gls{ide} Eclipse, make a common framework for many of the programming courses at NTNU, making them the main focus when looking for tools to support the students.
~\\%forklar hva eclipse og ide brukes til

\section{Research Questions}\label{intro-RQs}
~\\
The goal of this report is to look into the available tools that can aid developers, focusing on the potential use in the HMI-course.
This implies a focus on how to better understand the interactions within programs implementing the \gls{mvc} pattern, as well as the behavior of graphical interfaces.
Looking into the tools, it will identify the different types, as well as their strengths and weaknesses, using this as a basis for comparison.
After comparing, and determining which tool is likely to be of most use, the report will look at both the potential for improvements, and how it may fit into the teaching process, including the effect of any improvements made.
To be more specific, the following research questions have been formulated:
\begin{theorem}
What is the current state of the various visualization tools that are available?
\end{theorem}
\begin{theorem}
Could any of these be integrated into the current teaching environment at NTNU, consisting of Java and Eclipse? %slå sammen med 1?
\end{theorem}
\begin{theorem}
Is there room for improvement in how these tools are used, and the ease of using them? %presiser, nestten garantert nei nå
\end{theorem}
\begin{theorem}
Would the use of such tools and any improvements actually be useful for the students?%på hvilken måte?
\end{theorem}
