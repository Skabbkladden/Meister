\chapter{Methodology}\label{methodology}

While exploring the field in the previous chapter, the first research question was answered in the process.
The next three chapters will be used to answer the remaining questions using the following techniques, which are further elaborated on in the upcoming sections, which are:

\begin{itemize}
	\item{Review of existing tools}
	\item{An adaptation of the 'design science' process}%?
	\item{An evaluation with test subjects}
\end{itemize}

\section{Tool review}\label{methReview}

Having looked into the various tools available, and compared their feature sets to determine which one is most likely to fit with the current teaching environment, the second question is close to an answer as well.
In order to determine whether \gls{jive} can be properly integrated with the teaching environment, a closer study is necessary.
This study will focus on the features of \gls{jive}, looking at what information they give the users.
By examining the features, a fair assumption is that some of the most apparent issues will be revealed, providing a partial answer to research question three.

\section{Design science}\label{methDesign}

%find issues
%think of ideas
%determine possibility of implementation
%start implementation
	%encounter new  issues/challenges with impl
	%rework solution
%done


After thoroughly exploring \gls{jive}, and identifying potential improvements, it is time to select what to improve, and implement the necessary changes.
The available time-frame limits how many improvements it is possible to implement, as some time will most certainly be spent on understanding the source code of \gls{jive}.
Some time is also required at the end to answer the final research question.
This is best answered by performing a user evaluation with students in the target group.
This group consists of those described in the introduction, students in their second year of the computer science and informatics study-programs.

\section{Evaluation}\label{methEval}

In order to determine the usefulness of the selected tool, and the effect of the implemented changes, an evaluation with participants from the targeted user group is necessary.
The types of evaluations available are limited, with web-based questionnaires and in-person interviews being most suitable.

%web-based q
Reaching a wide audience, the use of web-based questionnaires can provide a large amount of data to support the research, but the nature of what is being evaluated makes this method unfit.
In order to give proper answers, the participants require a hands-on experience with the tool, and that would require them to acquire and install both the original version, and the modified version
This is a rather large obstacle that would deter most potential candidates from participating, and defeats the point of a questionnaire.
Although they could be tasked with comparing images of existing and new functionality, and given performance numbers to consider, it would not be the same as actually experiencing the differences, and getting to explore them at the users own pace.

Another issue appears in how questions are formulated, and what kind of answers they result in.
Requiring participants to quantify their experiences along numbered scales makes it easy to combine the results from all participants.
On the other hand, as each question must be limited in scope in order to get a precise answer, the amount of questions needed to get a good overview of each participants experience, can easily be perceived as different ways of asking the same question, resulting in repeated answers.

%interview
The alternative is an in-person evaluation with a smaller group of students, where the participants get access to a pre-configured system and enough time to make an informed opinion of what they are presented with.
As the availability of volunteers is limited, a properly executed in-person evaluation is likely to get more information out of the participants, by asking questions and discussing potential issues with them.

A hybrid approach could also be used by performing an in-person evaluation, and requiring participants to fill in a questionnaire afterwards.
This would still require one or more pre-configured systems, and some time spent on each participant to give an introduction to the tool they are evaluating.
Doing this in an effective manner, requires multiple assistants, and while the meaning of the questions could be further explained upon request, the answers would still be limited in the same manner as with a web-based questionnaire.


%design science, se på it3010 forskningsmetoder i informatikk