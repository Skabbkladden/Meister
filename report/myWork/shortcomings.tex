\section{Shortcomings of JIVE}\label{jiveShortcomings}

Even though JIVE offers several useful features, there is still room for improvements, both for general use, and more tailored towards development of graphical interfaces, as is the focus of the MMI-course.
~\\

The diagrams are possibly the most useful feature of JIVE, showing a lot of information in an understandable way, and offering some useful interaction.
But there is still room for improvement:
Inner types are displayed with their automatically generated, and fairly anonymous, name unless given a proper name in code.
By default, most of the classes defined by the JRE itself are ignored, omitting potentially important information in GUI-applications.
For the sequence-diagrams, it is naturally not important no show the internal behavior of every object, but they are also hidden in the contour-diagram.
Related to this, is the lack of visual connections between standard-objects, and for example instances of listeners added to them.
More related to the target group for this project, is a lack of differentiation between the different object-types in typical graphical architectures.
Especially MVC, whitch is a major focus of the MMI-course, has certain distinct types that are more important than others.
~\\

The sequence-diagrams can quickly become huge and hard to navigate, and the zooming function has very few levels outward, as well as leaving all text to small to read when zooming out.
Additionally there does not seem to be any way to vertically compact the sequence-diagrams, making them unnecessarily long in cases where calls to ignored or hidden methods are being made.
For instance when horizontally collapsing large parts of the diagram, leaving the parent lifeline at the same length as before, as can be seen in \autoref{fig:seqOving4CollapseB}.
Some of the papers on JIVE mention regex-folding,that could be used to substitute a series of events with a single new event, but it is still an experimental feature under development, and not available in the current release of JIVE.
~\\

Another potential for improvement is the process of stepping through recorded states, which currently requires manual interaction for each step, making complete replays of an execution infeasible due to the massive number of events recorded even for simple programs.
~\\

The search-view, while useful, is very strict in terms of what it looks for.
For example, searching for the creation of an object requires its full name, including packages, so searching for creations of JButton-instances would require a search for javax.swing.JButton.
~\\