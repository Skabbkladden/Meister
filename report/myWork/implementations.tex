\subsection{Implemented changes}\label{jiveImpl}%hva ble gjort, hvordan bruke funksjonene. Ikke så mye om hvordan. begrunne prioriteter?
~\\

%identifisering av instansierte grensesnitt, lambdaer og abstrakte klaser
The first change to be implemented was the identification and presentation of instantiated interfaces, which are now displayed with an appropriate icon, as well as being labeled with the interface it implements, instead of the class name that is assigned by default.
This function was also expanded to identify instances of abstract classes, and the lambda-expressions that were introduced with Java 8.
~\\

%utvidelse av filteret
The filtering function was expanded with the ability to specify packages that are not to be excluded from the execution model.
Adding a package to be included is as simple as adding a '+' in front of the package-name when adding it to the filter.
Unfortunately, a package may hold several classes that the user has no interest in seeing, but that are used by the few classes that the user is interested in.
This can result in very poor performance, and cluttered diagrams, but can be handled by adding the unwanted classes to the filter for exclusion.
~\\

%isolert visning av sekvens
The isolated view of the diagram does what was proposed in \autoref{fig:seqOving4IsolatedMock}: it displays the events caused by the selected event, and hides everything else.
By right clicking on an event in the regular sequence diagram, and selecting the 'Isolated view' option, the isolated view is triggered.
It is also possible to further focus on a part of the diagram from within the isolated view.
All of the functionality from the regular sequence diagram has been retained in the isolated view, so that the only difference is what parts of the execution are visible.
~\\
 
%mer avslappede søkekrav - er ikke utført for alle typer søk, er det verd å nevne i det hele tatt?
%hvorfor ble ikke alle forslag impementert? - tid for å bli kjent med jive, eclipse-plugins
