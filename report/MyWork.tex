\section{Enhancing JIVE}\label{enhJive}
%todo: referanser til artikler og ordliste, legg til figurer for å forklare bedre

In this section I will first explain the various features of JIVE, before identifying any shortcomings, and finally suggest improvements.

\subsection{Features}\label{jiveFeatures}

As mentioned in the prestudy, JIVE installs as a plugin in the eclipse IDE, adding another perspective in the environment.%figur, hele jive-perspektivet
The default views shown in this perspective are the "object diagram" and "sequence diagram" views, making the diagrams generated during debugging JIVEs two most apparent features.
The diagrams are updated according to the current state of the debugged program, so that the backtracking functionality allows you to see the entire execution graphically step-by-step.
The other views provided by JIVE are as follows: "contour model", "sequence model", "execution trace" and "search".
~\\

The object diagram-view shows the current state of the program by using a contour-diagram.%figur
Contour-diagrams are based on an old technique to  give semantics to Algol-like languages.
The basis has been extended to support modern concepts, such as object-oriented programming.
Objects are represented by a box, or contour.
Within the contour, the objects variables are shown, with name, type and value.
The contour also uses arrows to point at other contours that are related, e.g. an other object representing the value of a variable, or an enumerator.
Inheritance is shown by putting the contour of an object within the contour of the extended object. 
Object instances are kept separate from the contours representing inheritance, but will have relational links when necessary.
Method calls are also represented in the diagram, in their own contours, linked to the calling object.
JIVE offers to hide some of the information, such as inheritance or the composition of objects, in order to make the diagram smaller, and easier to read.
Something that can be especially useful when working with larger programs, with many objects and relations.
Visibility is aided by the use of colors to highlight specific elements of the diagram, such as variables bound to objects, and method-calls.
~\\

The sequence diagrams are fairly standard, with threads and objects represented by boxes on a row at the top, each with a vertical life-line.%figur
The actual sequence is shown with a thicker lifeline, with arrows representing method-calls.
In order to differentiate the threads where the execution is happening, the sequences are colored with the same color as the thread- box the sequence originates from, regardless of witch objects and methods are involved.
Right-clicking on a lifeline offers the ability to collapse method-calls originating from that lifeline in order to hide unnecessary information.
Right-clicking also offers to set the execution-state via the jump-option, updating the contour-diagram and -model to show that state.
~\\

Both of the diagrams can be saved as a high-resolution image at any time, so that one can look at diagrams from earlier runs, instead of being forced to view them through JIVE.
This also helps to visualize any changes made to a program, and to see what the effects are on the program flow.
They also share the ability to zoom in and out, further helping with the handling of larger diagrams.
~\\

Closely related to the diagrams, is the ability to quickly jump backwards and forwards between execution states.
This is enabled by the trace-log, containing an entry for every single event that occurs during execution of a program.
Each event is assigned an identifying number, in ascending order, and information about thread, type, caller, target, and the location of the source-code is stored.
The log is used as the basis for the models that make up the diagrams, and can be saved as both XML- and CSV-files for later use.
As mentioned in the Prestudy, logging every event has a significant impact on run-time performance, limiting the size of the programs that can be used in a meaningful way.
On the other hand, it allows the almost instant jumping between recorded states, as opposed to techniques that save a snapshot at predefined intervals, requiring the program to be run from the snapshot-state to the desired one, even if it is just a single step backwards.
~\\

The model-views each display an alternate view of their respective diagrams.%figur
They show the data-model representing the diagrams in a hierarchical structure, much like the organization of files and folders.
Right-clicking on an event in the sequence model allows you to set the execution state to that event, and have the diagrams updated accordingly.
Clicking on elements in the contour-model allows you to inspect the values of objects and their variables.
~\\

Finally, the trace-log enables the use of queries to search for specific events in the execution.
The queries are presented through a new tab in the eclipse search-window, easily accessible through the search-view, and comes with several pre-defined templates to simplify searching.
For example, searching for when a certain variable gets a certain value, only requires the user to specify the variable-name, its parent, and the value.
It is also possible to write custom queries, using the JIVE-Query-Language.
~\\

\subsection{Shortcomings of JIVE}\label{jiveShortcomings}

Even though JIVE offers several useful features, there is still room for improvements, both for general use, and more tailored towards development of graphical interfaces, as is the focus of the MMI-course.
~\\

The diagrams are possibly the most useful feature of JIVE, showing a lot of information in an understandable way, and offering some useful interaction.
But there is still room for improvement:
Inner types are displayed with their automatically generated, and fairly anonymous, name unless given a proper name in code.
By default, most of the classes defined by the JRE itself are ignored.
For the sequence-diagrams, it is naturally not important no show the internal behavior of every object, but they are also hidden in the contour-diagram.
Related to this, is the lack of visual connections between standard-objects, and for example instances of listeners added to them.
More related to the target group for this project, is a lack of differentiation between the different object-types in typical graphical architectures.
Especially MVC, whitch  is a major focus of the MMI-course, has certain distinct types that are more important than others.
~\\

The sequence-diagrams can quickly become huge and hard to navigate, and the zooming function has very few levels outward, as well as leaving all text to small to read when zooming out.
Additionally there does not seem to be any way to vertically compact the sequence-diagrams, making them unnecessarily long in certain cases.
For instance when horizontally collapsing large parts of the diagram, leaving the parent lifeline untouched.
Some of the papers on JIVE mention regex-folding, but it is nowhere to be found in the latest version of the plugin.
~\\

Another potential for improvement is the stepping through recorded states, which currently requires manual interaction for each step, making replays of an execution infeasible due to the massive number of events recorded even for simple programs.
~\\

The search-view, while useful, is very strict in terms of what it looks for.
For example, searching for the creation of an object requires its full name, including parent objects.
~\\

\subsection{Suggestions for improvement}\label{jiveSuggestions}

Based on the mentioned shortcomings, the following improvements are suggested.
~\\

The ability to detect an inner type with a generic name, and display its parent type in the diagrams instead of its own.
This will make, among others, listener-heavy programs much more understandable as one can see what kind of listener each object is, instead of guessing, based on when it is invoked in the sequence-diagram.%figur
Further helping the same situation, is to visually link listeners to the objects they are added to, in the contour-diagram, making it clearly visible which object is being listened to by the different listeners.%figur
Finding ways to highlight the different parts of e.g. an MVC-architecture, making it clearly visible which objects make out the models, views and controllers, may further help the understanding of a program.
But such highlighting must be balanced in order to not create a visual chaos of different colors.
The coloring, and naming of inner types should apply to both of the diagrams.
~\\