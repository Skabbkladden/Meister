%A
\newglossaryentry{agile}{name={Agile methods}, description={
A group of software development methodologies based on iterative and incremental development.\\
\url{http://en.wikipedia.org/wiki/Agile_software_development}}}

\newglossaryentry{synapse}{name={Apache Synapse}, description={
A lightweight and high-performance Enterprise Service Bus.\\
\url{http://synapse.apache.org/}}}
%B
\newglossaryentry{bandwidth}{name={bandwidth}, description={
Available or consumed data communication resources.\\
\url{https://secure.wikimedia.org/wikipedia/en/wiki/Bandwidth_(computing)}}}

\newglossaryentry{breakpoint}{name={breakpoint}, description={
A source code marker telling the debugger to halt program execution at a certain point.}}

\newglossaryentry{broker}{name={Broker} , description={
Our middleware layer works as a QoS broker for services and clients.
Broker as referred to by the report given to us from FFI: a centralized ‘server’ of sorts which gathered Bandwidth data from tactical routers.}}

%C
\newglossaryentry{cli}{name={CLI}, description={
Command Line Interface, a text-based interface for interacting with programs via a terminal.}}

\newglossaryentry{codecanvas}{name={Code Canvas}, description={
A visualization-tool for Microsoft visual studio, showing code, diagrams and documents on a large layered canvas.\\
\url{http://research.microsoft.com/en-us/projects/codecanvas/}}}

\newglossaryentry{contourdiagram}{name={contour diagram}, description={
An enhanced object diagram, showing objects, their variables and their relations to other objects.\\
\cite{Jayaraman1996, Streib2010}}}%TODO:refer to paper

\newglossaryentry{cots}{name={COTS} , description={
Commercially available Off-The-Shelf often used to talk about services which the customer wants to use server side.\\ \url{https://secure.wikimedia.org/wikipedia/en/wiki/Commercial_off-the-shelf}}}

\newglossaryentry{credentials}{name={credentials} , description={
User-supplied credentials in the form of a username, password, role triplet.}}

%D
\newglossaryentry{debugVisualizationPlugin}{name={Debug Visualization Plugin}, description={
An Eclipse plugin that provides a graphical view of variables during debugging.\\
\url{https://code.google.com/a/eclipselabs.org/p/debugvisualisation/}}}

\newglossaryentry{diffserv}{name={DiffServ} , description={
Differentiated services, a field in the IPv4 header.\\
\url{http://www.networksorcery.com/enp/rfc/rfc2474.txt}}}

%E
\newglossaryentry{esb}{name={ESB} , description={
A software architecture model used for designing and the interaction and communication between mutually interacting software applications.\\ \url{http://en.wikipedia.org/wiki/Enterprise_service_bus}}}

\newglossaryentry{executiontrace}{name={execution trace} ,description={
A log of all changes to the state of a program throughout its execution.}}

%G
\newglossaryentry{gantt}{name={Gantt Chart}, description={
A type of bar chart that illustrates a project schedule.\\
\url{http://en.wikipedia.org/wiki/Gantt}}}

\newglossaryentry{gdb}{name={GDB}, description={
GNU debugger. A multiplatform, multilanguage CLI-debugger with tracing.\\
\url{http://www.sourceware.org/gdb/}}}

\newglossaryentry{git}{name={Git} , description={
A free and open source, distributed version control system.\\
\url{http://www.git-scm.com}}}

\newglossaryentry{github}{name={GitHub} , description={
A web-based hosting service for software development projects that use the Git version control system.\\
\url{http://www.github.com}}}

\newglossaryentry{glassfish}{name={GlassFish} , description={
An application server written in Java.\\
\url{http://glassfish.java.net/}}}

%H
\newglossaryentry{http}{name={HTTP}, description={
Hypertext Transfer Protocol. The foundation of data communication on the World Wide Web.\\
\url{http://www.w3.org/History/19921103-hypertext/hypertext/WWW/Protocols/HTTP/AsImplemented.html}}}

\newglossaryentry{httpcomponents}{name={HTTPComponents}, description={
A toolset of low level Java components focused on HTTP and associating protocols.\\
\url{http://hc.apache.org/}}}

\newglossaryentry{httpCore}{name={HttpCore}, description={
HttpCore is a set of low level HTTP transport components that can be used to build custom client and server side HTTP services with a minimal footprint. It is a component of the \gls{httpcomponents} package.}}

%I
\newglossaryentry{ide}{name={IDE} , description={
Integrated Development Environment. A software application that provides facilities for software development such as source code editor, compiler etc.}}

\newglossaryentry{identity server}{name={Identity Server} , description={
\url{http://wso2.com/products/identity-server/}}}

\newglossaryentry{ipaddress}{name={IP address}, description={
A numerical label assigned to each device connected to the Internet.}}

\newglossaryentry{ip header}{name={IP header} , description={
The header of an IPv4 packet.}}

%J
\newglossaryentry{java}{name={Java} , description={
An object oriented and cross platform programming language.\\
\url{http://www.oracle.com/us/technologies/java/overview/index.html}}}

\newglossaryentry{java coding conventions}{name={Java Coding Conventions} , description={
\url{http://www.oracle.com/technetwork/java/codeconv-138413.html}}}

\newglossaryentry{javavis}{name={JAVAVIS}, description={
Tool that generates UML diagrams from running java applications.\\
\cite{Oechsle2002}}}%todo:url

\newglossaryentry{jinsight}{name={Jinsight}, description={
An advanced debugger made by IBM, supports visualization, and powerful analysis.\\
\cite{Pauw}}}%TODO: url

\newglossaryentry{jive}{name={JIVE}, description={
An advanced debugging tool supporting visualisation, backward stepping, and querying.\\
\url{http://www.cse.buffalo.edu/jive/}}}%TODO: refer to papers?

\newglossaryentry{jql}{name={JQL}, description={
Jive Query Language, used to formuate queries within the jive debugging environment}}%TODO: paper/ref

\newglossaryentry{junit}{name={JUnit} , description={
A testing framework for the Java programming language.\\
\url{http://junit.org/}}}

%L
\newglossaryentry{latex}{name={\LaTeX} , description={
A document preparation system for the \TeX typesetting program.\\
\url{http://www.latex-project.org/}}}

\newglossaryentry{LXC}{name={LXC} , description={
Linux Containers.\\
\url{http://lxc.sourceforge.net/}}}

%M
\newglossaryentry{maven}{name={Maven}, description={
Apache Maven is a software project management and comprehension tool. Based on the concept of a project object model (POM), Maven can manage a project's build, reporting and documentation from a central piece of information.\\
\url{https://maven.apache.org/}}}

\newglossaryentry{mediator}{name={mediator} , description={
A component in WSO2 ESB which can be used to work on incoming or outgoing messages that passes through the ESB.\\
\url{http://synapse.apache.org/Synapse_QuickStart.html}}}

\newglossaryentry{message}{name={message} , description={
SOAP message.\\
\url{https://secure.wikimedia.org/wikipedia/en/wiki/SOAP\#Message_format}}}

\newglossaryentry{message context}{name={message context} , description={
Component in the ESB, contains the message, as well as all information about it, including network sockets.\\ \url{http://synapse.apache.org/apidocs/org/apache/synapse/MessageContext.htm}}}

\newglossaryentry{middleware}{name={middleware} , description={
In the report middleware will refer to the program we are making. Other distinctions should be made explicitly in the text.}}

\newglossaryentry{MobiEmu}{name={MobiEmu} , description={
Mobility Emulator, A framework for emulating mobile ad-hoc networks with Linux containers and ns-3.}}

\newglossaryentry{ms}{name={MS} , description={
Please see \Gls{monitoring service}.}}

\newglossaryentry{monitoring service}{name={Monitoring Service}, description={
Monitoring Service, a service that provides bandwidth monitoring, running on the same server as the Tactical Router.}}

%N
\newglossaryentry{ns-3}{name={ns-3} , description={
A network simulator.\\
\url{http://www.nsnam.org/}}}

%O
\newglossaryentry{opensaml}{name={OpenSAML} , description={
A set of open source C++ \& Java libraries to support developers working with SAML.\\
\url{https://wiki.shibboleth.net/confluence/display/OpenSAML/Home/}}}

%P
\newglossaryentry{packet}{name={packet} , description={
IP packet refers to the format to which a data transmitted over the IP protocol has been formatted to.\\
\url{http://en.wikipedia.org/wiki/IPv4\#Packet_structure}}}

\newglossaryentry{packet sniffer}{name={packet sniffer} , description={
A packet sniffer is a computer program or a piece of computer hardware that can intercept and log traffic passing over a digital network or part of a network.\\ \url{https://en.wikipedia.org/wiki/Packet_analyzer}}}

\newglossaryentry{pcap}{name={pcap}, description={
pcap is short for Packet capture which in our text this usually refers to a program which captures the traffic on a given socket.\\
\url{https://secure.wikimedia.org/wikipedia/en/wiki/Pcap}}}

\newglossaryentry{proxy}{name={proxy}, description={
A proxy server is a server that acts as an intermediary for requests from clients seeking resources from other servers.\\
\url{http://en.wikipedia.org/wiki/Proxy_server}}}

%Q
\newglossaryentry{qos}{name={QoS} , description={
Please see \Gls{quality of service}.}}

\newglossaryentry{quality of service}{name={Quality of Service}, description={
Quality of Service refers to several related aspects of telephony and computer networks that allow the transport of traffic with special requirements.\\ \url{http://en.wikipedia.org/wiki/Quality_of_service}}}

%S
\newglossaryentry{saml}{name={SAML}, description={
Security Assertion Markup Language.\\
\url{https://secure.wikimedia.org/wikipedia/en/wiki/SAML}}}

\newglossaryentry{scrum}{name={Scrum}, description={
An agile software development methodology.\\
\url{http://en.wikipedia.org/wiki/Scrum_(development)}}}

\newglossaryentry{serviceprovider}{name={Service Provider}, description={
An entity that provides Web services.}}

\newglossaryentry{soap}{name={SOAP} , description={
A lightweight protocol intended for exchanging structured information in the implementation of Web services in computer networks.\\
\url{http://www.w3.org/TR/soap12-part1/\#intro}}}

\newglossaryentry{svn}{name={Subversion} , description={
Subversion exists to be universally recognized and adopted as an open-source, centralized version control system characterized by its reliability as a safe haven for valuable data; the simplicity of its model and usage; and its ability to support the needs of a wide variety of users and projects, from individuals to large-scale enterprise operations.\\
\url{https://subversion.apache.org/}}}

%T
\newglossaryentry{tactical router}{name={Tactical Router} , description={
A Multi-topology router used in military networks.}}

\newglossaryentry{tod}{name={Trace-Oriented Debugger}, description={
Trace-Oriented Debugger. A debugging tool that executes queries on program traces.\\
\cite{Pothier2007}}}

\newglossaryentry{token}{name={token} , description={
 A SAML token from some form of identity server, possibly with additional meta data.}}

\newglossaryentry{tos}{name={TOS} , description={
Type of Service, a field in the IPv4 header, now obsolete and replaced by diffserv.\\
\url{http://en.wikipedia.org/wiki/Type_of_Service}}}

\newglossaryentry{tr}{name={TR}, description={
Please see \Gls{tactical router}.}}

\newglossaryentry{traceviewer}{name={Trace Viewer Plugin}, description={
Tool to visualize and analyze communication of parallel message passing programs.\\
\cite{Kranzlmuller}}}

%W
\newglossaryentry{waterfall}{name={Waterfall model}, description={
A sequential design process often used in software development, in which development is supposed to proceed linearly through the phases of requirements analysis, design, implementation etc.\\
\url{http://en.wikipedia.org/wiki/Waterfall_development}}}

\newglossaryentry{wbs}{name={WBS}, description={
Work Breakdown Structure. An oriented decomposition of a project into smaller components.\\
\url{http://en.wikipedia.org/wiki/Work_breakdown_structure}}}

\newglossaryentry{webservice}{name={Web service}, description={
A software system designed to support interoperable machine-to-machine interaction over a network.\\
\url{http://www.w3.org/TR/2004/NOTE-ws-gloss-20040211/\#soapmessage}}}

\newglossaryentry{ws-security}{name={WS-Security} , description={
An extension to SOAP to apply security to Web services}}

\newglossaryentry{wso2 esb}{name={WSO2 ESB} , description={
An Enterprise Service Bus built on top of Apache Synapse.\\
\url{http://wso2.com/products/enterprise-service-bus/}}}

\newglossaryentry{whyline}{name={Whyline}, description={
A query-based debugger that provides an easy way to find out why things are as they are.\\
\cite{ko2009}}}

%X
\newglossaryentry{xacml}{name={XACML} , description={
eXtensible Access Control Markup Language.\\
\url{https://secure.wikimedia.org/wikipedia/en/wiki/Xacml}}}

\newglossaryentry{xml}{name={XML}, description={
eXtensible Markup Language. A markup language defining a set of rules for encoding documents in a format readable for both humans and machines. \url{http://www.w3.org/TR/REC-xml/}}}

\newglossaryentry{xp}{name={XP}, description={
Extreme programming is a type of agile software development.\\
\url{http://en.wikipedia.org/wiki/Extreme_programming_practices}}}


