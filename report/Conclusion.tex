\section{Conclusion}\label{conclusion}
This section will summarize the previous sections in light of the research questions proposed in the introduction.

%resultat av undersøkelse
%svar på RQ
\subsection{Research Questions}\label{conclusionRQs}


\begin{theorem}
What is the current state of the various visualization tools that are available?
\end{theorem}

There are currently several different tools that can aid a developer in his understanding of programming, ranging from the regular debuggers found in most IDEs, to code analysis and execution tracers.
Most of these focus on one or two features, instead of attempting to combine them all.
Regarding NTNUs chosen teaching environment, the amount of available tools is naturally reduced, but there is still a large portion of tools available, thanks to the plug-in support of Eclipse.
Among these, JIVE was found to be of most interest, due to its combination of features, including tracing, diagram generation, and state-jumping.
~\\

\begin{theorem}
Could any of these be integrated into the current teaching environment at NTNU, consisting of Java and Eclipse?
\end{theorem}

Both JIVE and other tools are available as Eclipse-plugins, and as such, they can easily be integrated into the courses that use Eclipse.
Unless a very platform-specific mandatory framework is used, students are of course free to choose a different IDE while following  a course, and by doing so, they will also opt out of the use of JIVE or any other tools.
But these students are not likely to be the ones that would benefit the most from such tools in the first place, as a decision to not use recommended tools usually indicates that they know what they are doing. %? ikke akkurat noe jeg har undersøkt, bør ha mer for å støtte opp om dette
As mentioned in some of their papers, JIVE has already been successfully used in courses at Buffalo University.%REF!!
~\\

\begin{theorem}
Is there room for improvement in how these tools are used, and the ease of using them?
\end{theorem}

As detailed in \autoref{jiveEnhance}, there are definitely a potential for improvements in how JIVE works, and is used.
While the existing features are useful as they are, they were found to be lacking in usability.
The diagrams were found to lacking in their description of certain classes, making identification difficult.
Larger programs were also found to quickly grow the sequence diagram to large sizes, making it harder to get an overview.
~\\

Some of these points were the focus of the improvements that were implemented in \autoref{jiveImpl}.
~\\

\begin{theorem}
Would the use of such tools and any improvements actually be useful for the students?
\end{theorem}
~\\

%framtidig arbeid
\subsection{Future work}\label{conclusionFuture}
