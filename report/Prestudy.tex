\section{Prestudy}\label{Prestudy}
~\\

Methods:\\

Visualization:\\
	Generating graphs and diagrams representing the program\\
	Easier to get an overview of program structure and execution\\

Interactive forwards- and backwards-stepping\\
	two forms: re-execution, state-saving\\
	re-execution: small memory footprint,  slow backward stepping\\
	state-save: fast stepping both ways, needs more memory, amount depending on program\\

Queries:\\
	fast way to check object-relations and -properties\\

Tools:\\

GNU debugger (GDB)\\
	tracing, reverse debugging, general debug-stuff\\
	multiplatform, multi-language\\
	remote debugging\\
	CLI-only, needs separate front-end\\

Jinsight\\
	made by IBM\\
	two components: profiler and visualizer\\
	only for z/OS or Linux on system z\\
	builds a trace when application is running\\
	client connects to profiler and visualizes the trace\\
	modified JVM?\\
	120 minute trace limit\\
	very powerful\\

Javavis\\
	relies on the Java Debug Interface (JDI), and the Vivaldi Kernel (a visualization library)\\
	shows dynamic behavior of running program\\
	object diagrams+sequence diagram, UML\\
	smooth transitions\\
	not a debugger\\
	
code canvas (visual studio)\\
	unites all project-files on a infinite zoomable surface\\
	both content and info\\
	layers of visualization - files/folders, diagrams, tests, editors, traces ++\\
	several layers visible at the same time\\
	search\\

trace viewer plugin (g-Eclipse)\\
	g-eclipse=grid, archived project\\
	visualize and analyze communication of message-passing programs\\
	standalone/platform independent\\
	designed for massive parallelism\\
	debugging\\
	event markers\\

Whyline\\
	Interrogative debugger\\
	why did, why did not\\
	works on recorded executions\\

TOD: Trace-Oriented Debugger\\
	omniscient debugger\\
	queries\\
	dynamic visualizations - high-level, graph of event density\\

Jive\\
	combines ale fields\\
	contour diagram\\
	sequence diagram\\
	stepping - state-saving\\
	queries - enabled by state-saving\\
	can be used for debugging\\




