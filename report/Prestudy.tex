\chapter{Prestudy}\label{prestudy}%veldig rett på, forklar mer om ulike typer diagram og metoder for å visualisere, hva skiller disse fra debuggere?
This chapter will serve to get an overview of the current situation of debugging methods, visualization techniques, and the various tools that are available.
\section{Methods}\label{preMethods}
There are several ways a debugger can aid the programmer beyond just showing the current state of a program at a \gls{breakpoint} chosen before running.
For a fresh programmer, either in general, or at a certain project, the most useful method is probably to generate diagrams that visualize the current state, and the path of execution.
I.e. some form of object-, and/or sequence-diagram.
Such diagrams can make it easier to get an overview of a programs current state, see the contents of objects and how they relate to each other, and to understand how the various components work together.
~\\

Object diagrams are similar to class diagrams, in that they both show the objects of a program, and their connections between each other.
Where they differ, is that while the class diagram shows the connection between classes, and what the classes are composed of, the object diagram shows the state of a program at a certain point in time.
As a program is executed, its state changes, and the exact combination of objects and the connections between them will change accordingly, and the object diagram reflects this.
Some programs will return to a certain state after performing a task, while others do not have this `stable state'.
The transition between stable and working states can be illustrated with a state diagram, which typically abstracts the states to `idle' and various `tasks', like `processing input'.
A sequence diagram shows the order in which the program is executed, showing components invoking methods on each other.
~\\

Depending on the desired kind of diagram, different techniques are used for the generation.
Class diagrams can be made by analyzing the source code of a program, while object diagrams require information that can only be acquired by analyzing a running program, and logging what happens.
Such a log, or \gls{executiontrace}, can also be used to create a sequence diagram, as those also need information that can not necessarily be acquired by analyzing code.
State diagrams are usually created manually by a system architect, as automated identification of states is not a trivial task.
~\\

Execution traces can also be used to enable backwards stepping of program execution.
Stepping back in time allows the user to not only see the failure state of a program, but to go back and see what caused the problem, instead of adding a new \gls{breakpoint} and running the program again.
There are different ways to store the information describing each step, providing various trade-offs between access-time and memory usage.
The straightforward way would be to store each state independently, sacrificing memory for a near-constant time to load the details of a step.
The load time will naturally be affected by the amount of data that must be analyzed, but there is no need to work through any the steps in between.
If the data is stored as a differential of the previous state, the opposite situation is created.
The amount of memory required is significantly reduced, and stepping to the immediate neighbors is very fast, but jumping between any two steps would require analyzing every step in between the two.
A hybrid approach is also possible, relying on a differential model, but also introducing checkpoints every n steps, where all information is stored.
This will provide a balance between the two previous methods, and could be adjusted to fit the characteristics of the system the tool is running on.
~\\

One can avoid the potential disadvantages of manual backstepping by using queries instead.
Queries enable the user to asks the debugger about the current and earlier states of execution in a simple way.
The debugger then does the work of finding what was asked for, instead of the user manually searching through the program states.
~\\

A major downside of execution traces, is the performance penalty from doing extra work for every step in the execution.
The amount of work will vary, depending on implementation details and of course what is done with the log.
If all the tracing-process does is write to a log-file, which will be used later, one can achieve significantly smaller performance impact compared to a system that does real-time analysis of the data.
~\\

%-----------------------------------------------------------------------------------------------------------------------------
\section{Existing Tools}\label{PreTools}%mer om hvilke metoder som blir brukt av hvert verktøy
There currently exists several tools that provide one or more of the methods mentioned above.
The following list is not extensive, but includes some of the most interesting tools, considering the focus of this report.
Some of the tools listed below does not support Java or eclipse, but provide interesting features that are worth mentioning, despite their incompatibilities with the desired teaching environment.
Download links for the tools are provided in their respective glossary entries for those tools that are available.
~\\

%not java/eclipse
\gls{gdb}, a part of the GNU project /FSF?,  offers a tracing environment, and support for many languages, but not Java.
Due to its \gls{cli}, it is not necessarily easy to use on its own, and so, there are several front-end platforms that provides a graphical environment around GDB, including Eclipse.
~\\

\Gls{codecanvas} \cite{Deline2010}, developed by Microsoft, uses an interesting way of visualizing an entire project, everything from source-code to design documents and diagrams are layered onto a large canvas, allowing easy navigation between various elements, but is restricted to Microsoft Visual Studio, and the languages it supports.
~\\

%java/eclipse
The \gls{traceviewer} \cite{Kranzlmuller} for g-eclipse -- a now discontinued version of eclipse, geared for development of grid-computing software -- uses a trace to generate visualizations of the program execution, and thus makes it easier to understand, but is designed for massive parallelism, requires a special version of eclipse, and may not be very useful for understanding smaller programs.
The plugin was developed by the MNM-team at the Ludwig Maximilian University of Munich.
~\\

The \gls{tod} \cite{Pothier2007}, is developed at the University of Chile in Santiago, and as the name suggests, utilizes execution traces to enable its debugging and querying features.
It offers  high performance tracing, being able to maintain usable interaction while debugging complex software, but its only visual representation of programs is the 'mural', a graph that shows event density over time.
~\\

\Gls{whyline} \cite{ko2009}, like the \gls{tod}, makes use of execution traces to enable querying, instead of providing visualizations.
It aims to explain why something happened in a program, and does so by looking at the history of the involved components.
This tool only exists as a separate application, and does not integrate into any \gls{ide}.
\Gls{whyline} is developed at the Carnegie Mellon University.
~\\

The \gls{debugVisualizationPlugin} for eclipse provides an alternative to the variable view provided in eclipses debug-perspective, providing a graph that represents the variables of a program.
The user still needs to use regular techniques they would use normally in order to pause the program-execution, and be able to actually view the state of the variables.
This was previously tested in the course TDT4100~-~Object oriented programming, with the conclusion that the generated diagram quickly became too large, showing the contents of objects that were not important, and shuffling objects as new ones were added.%hvem står bak?
~\\

\Gls{javavis} \cite{Oechsle2002},developed at the University of Applied Sciences in Trier, provides visualizations in the form of \gls{uml} sequence- and object-diagrams, but does not provide any debugging features.
~\\

\Gls{jinsight} \cite{Pauw} is a powerful tool built by IBM, supporting both tracing and visualization.
However, it is restricted to z/OS and linux on system Z, preventing most people from using it.
~\\

\Gls{jive} is a tool that utilizes execution traces to provide diagrams while running a program.
Developed at the university of Buffalo, it is installed a an Eclipse plugin, and provides several new views to display the information it provides.
In addition to providing diagrams, the trace log is also used to enable backstepping, which is coupled with the diagrams to always show the selected execution state.
~\\

%-----------------------------------------------------------------------------------------------------------------------------
\section{Deciding on a tool}\label{preDiscuss}
%traceviewer, debug viualization plugin, jive, tod er eclipse plugins
Of the mentioned tools, the \gls{traceviewer}, \gls{tod}, \gls{debugVisualizationPlugin}, and \gls{jive} are all available as eclipse-plugins, and are thus fairly simple to integrate into the existing teaching process.
The features they provide, on the other hand, vary.
The \gls{traceviewer} is as mentioned, designed for massive parallelism, and does require a specific version of eclipse, and because of this, it is not really suited for further study.%?
The \gls{tod} provides a debugging environment supported by trace logs, as implied by the name.
It presents its information mostly in a textual way, and does not provide any visual diagrams of the program structure, or execution order.
The \gls{debugVisualizationPlugin} expands on the debugging functionality of Eclipse by providing a visual view of variables, and is designed to be used alongside the rest of the debugging environment provided by eclipse.
~\\

%stuff about jive coming up, sloppypar to fix cite going over margins
\begin{sloppypar}
\Gls{jive} seems to be the only tool that utilizes all three methods mentioned in \autoref{preMethods}, as well as being freely available as a plugin for eclipse, making it easy to install and use.
During program execution, Jive generates a \gls{contourdiagram} \cite{Jayaraman1996}, and a sequence diagram.
Combined with an execution trace, it allows the user to jump back and forth in the execution, and have the diagrams updated accordingly.
Querying is supported with pre-defined search-templates added to the built-in search window in Eclipse.
~\\
\end{sloppypar}

Due to all the extra work being done when using jive to debug a program, the performance is not always acceptable.
For small non-interactive programs, the added waiting time may not be a problem, but larger programs are likely to suffer from a significantly longer execution time, and even simple interactive programs can use up to a second to respond to input on a fairly powerful computer.
~\\





